\chapter{Related Work}
\section{Applications - Tilt Brush}
Since the release of both the Oculus rift [REF] and HTC Vive [REF] which both have 6 DOF controllers, the number of applications focusing on complex interactions and manipulation has really exploded. One of the most known application is 'Tilt Brush' [REF] for HTC Vive where the user can use the existing handcontrollers as paintbrushes and paint in 3D space. This application highlights many of the interactions that are esential when manipulating a the environment. Some of these are listed below:
\subsection{Creating objects}
Creating an object or in this case a new painting stroke is the core of the application, and can as such be accessed directly with primary triggers on both controllers.
\subsection{Selecting and deleting objects}
Objects (strokes) are targeted by placing a spherical tool (attached to your brush) around the stroke and pressing the same trigger that is used for drawing. The tool is activated by selection on the secondary brush menu.
\subsection{Parameters and setting}
The most common and used tools for what will be painted ( brush-size, undo, deleting strokes etc) are accessed through the primary brush (controller). More complex modifications (brush type, colors etc) are accessed from the secondary brush by activating a menu connected to that brush (controller) and selecting with the other brush.
[*****ADD IMAGE OF BRUSH MENU*****]
\subsection{Scale and positioning}
One of the biggest advantages of working in a virtual environment is that you are not bound to the restrictions of your physical relatioonship to the objects that you are working with. By using both controllers and moving them away from eachother, the environment grows and the users scale decreases. The user scale can be enlarged by moving the controllers closer together.
By using a raycast (see Section \ref{theory:toolsandtech}) the user can teleport around in the virtual environment.
\section{Research and concepts}
An environment with a purpose similar to the objective of this thesis has been researched and developed by Bowman et al. back in 1997. \cite{relatedwork:kijimaand1997transition} The authors investigated how to combine a regular work setup for 3D development with a HMD to remove the repeated transition between the two. Their Augmented Reality (AR) system is based on a real workstation which consists of a computer together with screen, keyboard and mouse. This is combined with a Projective Head Display (PHMD) to create an environment where the user can see projected 3D objects and still be able to use the workstation as usual.    
