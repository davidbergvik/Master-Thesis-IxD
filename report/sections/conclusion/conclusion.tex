\chapter{Conclusion}
This study highlights important issues in the current process of designing experiences for virtual reality. The gap between the design medium (desktop computer or laptop) and the target virtual environment (using a HMD) creates frictions for professionals within this field of work. Having to switch between these two mediums is both timeconsuming and causes an unnesisary amount of errors, which correlates with the effects from increased amount of cognitive load. Furthermore, a tool like the one explained in this study is the prevents the computer-to-VR transitions that otherwise exists, which can prevent 'Flow'.

Based on the experience design process and the VR technology available today, this study has distinguished that utilizing VR as tool is valid in certain phases of the process. Two fundamental tool have been identified, designed and tested based on these findings; In the early stages of the experience design process a tool for rough prototyping provides an intuitive and fast way to explore concepts without requiring previous scripting or VR development knowledge. When the concepts have been selected and are being developed, a tool for object and scene manipulation of the actual concept in real-time (using Unity EditorVR or otherwise) is preferable.
