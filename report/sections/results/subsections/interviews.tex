\section{Findings from interviews}
\label{result:interviews}
A lot of aspects that arised from the interviews are referenced in the Theoretical Framework (see Chapter \ref{theory}), therefore these will not be showcased in this section. For details about the content of all interviews, see Appendix \ref{appendix:interview:answers}.  There were however answers and insights that did not appear before the interviews and were at most speculations. These will be summarized in this section.
\subsection{The learning-curve and programming knowledge}
\label{result:interviews:learningcurve}
The majority of insights in this section are gathered from question 7: \textquote{What are the biggest problems you meet in your daily work?}. One of these insights were that in many interviews was that in order to work with a VR project as a designer/UX/developer or art director some level of programming knowledge is required. The reason is that in order to create or change something in a developed VE, this has to be done in a 3D game engine such as Unity\footnote{https://unity3d.com/} and Unreal\footnote{https://www.unrealengine.com/what-is-unreal-engine-4}. This has been mentioned previously in this study, but the ramafications and the differences of why this is an issue was not. From the interviews these differences can be divided into two parts, prototyping (read more in section \ref{method:prototype})/concept creation and testing/fine tuning.

\textbf{The first part} consist of trying a concept or idea in VR on the fly, from something like a paper prototype. At the moment this requires a lot of knowlegde of the software that the application is developed in, which is troublesome if you're not a developer and without knowledge of this software. A UX designer that needs to test a UI interaction or animation can not effectivly use the same tools that is used for  screen designs. A tool for simple rough creation and manipulation requested in order to bring all types of designers up to speed in prototyping for VR. For a team in an early concept phase it's important to get the explore priorities and finding a balance between features and budget. By creating concepts early on, these estimates can be evaluted by creating rough prototypes in VR and analyzing complexity, resources any problems that can occur for each feature.

 For \textbf{the second part}, the medium is slightly different. When the software is in the production-phase, according to interviews with members of a VR core team, a lot of time is assigned to tweaking the existing VE and manipulating objects from the users' perspective. This is troublesome when the adjustments are made on a computer, and the testing is done with a HMD. Interviewees explains how the constant switching between the two medium is both time-consuming and painful in the long-run, as the HMD creates chafings on their face and forehead from sliding the HMD on and off between tests. The optimal solution to this problem is explained as a tool that can be used with a HMD to tweak properties of the VE when immersed.
 \subsection{Time-consuming processes}
 When asked about the time and effort of each parts in their design process, there were two major issues across the board. \textbf{Visiulizing and explaining the design} is really hard until it is created in VR. Furthermore the effect of this is a creative gap between the brainstorming (creating concepts) and the implemented design, it also limits the collaboration between team-members with and without programming experience. \textbf{Switchingh between mediums} when designing is inefficient aswell as physically painful. As the current process is explained in two steps; \begin{enumerate}
   \item Creation and manipulation (on a desktop computer)
   \item Testing and validation (with a HMD in VR)
 \end{enumerate}
Switching between mediums like this significally extends the designing and development phase, and it keeps them out of their "zone". This is supported by the means to maintain Flow (read more in section \ref{theory:user-centered:flow}) as the user gets no feedback from their actions (eg. moving or creating an object, they have to go into VR to see the results in the correct medium). It also interupts the user from creating into a state of pure observing and testing.
\subsection{Field-specific issues}
When asked about flaws within the current design process and what features that a VR tool should consist of, there were similarities as well as diversity within the responses. By compiling all responses, the result display as a venn diagram (Figure \ref{fig:result:interview:venn}) consisting of the two parts that were explained in section \ref{result:interviews:learningcurve}. This provides a overview of what is expected for each part of the process. It can be argued that this segregation should push for a two level solution for a VR tool.
\subsection{Design Process and future}
The design processes that were described have significant differences between parts of the industry, for the experience design teams it still consists of a lot of experimentation and concepts trials. According to the interviees at North Kingdom, using a "Previz" which is a rough version of the experience, allowd the team to communicate their design ideas and feelings to a client. Several interviewees mentioned that one of the biggest uses for an immersive tool is when creating and modifying the previz. This would allow them to create and manipulate the previz in VR until they are satisfied, ship the previz and later build the final implementation with the previz as a foundation.

When asked about the future and what they look forward to seeing for VR development, the number one this across the board was the ability to make changes and test when immerged in the project that they are working on.

\subsection{Features}
None of the interviewees had tried a tool scene creation before, only creative applications like Tilt Brush (details in section \ref{theory:best-practice}). They did however mention features that they expecxt or require in order for them to work in VR. Ergonomically and practically, the concensus were that this tool should be usable from a sitting position. This would allow them to work from there current workspaces today and would be less exhausting over time. This correlates with previous research about human body limitations and VR, which is summarized in section \ref{theory:interactionissues}. Below is a list of features that members of experience-design core teams would prefer to have in a design tool for VR projects:
\begin{itemize}
  \item Overall
  \begin{itemize}
    \item Access objects from a library and create a scene
    \item Manipulate objects
    \item changing object parameters inside VR
    \item Repeat animations while tweaking timings and distances
  \end{itemize}
  \item Prototyping
  \begin{itemize}
    \item Creating rough animations
    \item Try out interactions and movements
    \item Adding and adjusting sound and lighting
  \end{itemize}
  \item Development
  \begin{itemize}
    \item Place and change details (tweak) things in VE.
  \end{itemize}
\end{itemize}
Apart from this, some parts where added as "not preferrable". Most of these are part of the development phase, and has a significant amount of scripts involved. According to the interviewees these are better suitable for a dektop computer. From this, a venn-diagram was created to categorize the features based on the phase and title of the user.
