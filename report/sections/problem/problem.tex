\chapter{Problem}

\section{North Kingdom}
This thesis will be carried out together with a company called North Kingdom. The company defines itself as an "experience design company", founded in Skellefte and has been a public company since 2003. Today, the offices of North Kingdom are in Skellefte\aa, Stockholm and Los Angeles. \cite{northkingdom} Their vision is (from their website)
\begin{quote}
We Believe That new value can be created wherever people, business, and technology collide. We help our clients harness That value through the creation of experiences, products, and services That play a meaningful role in people's lives. Through human-centered design, we make the complex simple and relatable, no matter what medium or platform. \cite{northkingdom}
\end{quote}
This thesis will run in parallel with North Kingdom's current VR projects. Their design process and pipeline will act as the scope for this thesis.

\section{Purpose}
As North Kingdom shifts their focus towards creating experiences in virtual reality, their design process has to be modified. As a part of that, this thesis will evalute the oppertunity of encorperating a VR editor into their new VR design process. This could ease the fact that designing 3D environments and experiences is highly dependent on the actual perspectives and positions of the end user, and when designing these environments on a screen some aspects of the design becomes hard to predict.

\section{Goals}
By allowing the designer and developer to manipulate this environment when emerged might solve some of these issues and create a more intuitive user experience during the creation and manipulation phase. It can allow users to make instant tweaks to the world without having to leave it. I will also evaluate if this tool can be used for rapid prototyping and bridge the gap between team-members by letting non-developers edit the prototype to explain their ideas.


\begin{itemize}
	\item Study how VR projects are carried today and identify opportunities where immersion is benifitial
	\item Design VR interfaces that are user-friendly and task specific from the results of the study.
	\item Build functional prototypes based on the final design
	\item These prototypes should be evaluated
\end{itemize}

\section{Limitations}
This thesis will be based on the current technology and software on the market. The concepts and prototypes created will be restricted to the posibilities of what can be implemented into a working system today. The scope of what will be investigated is restricted to the field of VR experience and experience design along with its implementations.

\section{Terminology}
\begin{tabular}{ | p{5.5cm} | p{6.5cm} |  }
	\hline
     	\multicolumn{2}{|c|}{Common terms and their explanation} \\
     	\hline
	Field of View (FOV) & What a user can see without turning head or body \\
    	\hline
     	Head-Mounted Display (HMD) & A display device worn on the users head with a small screen in front of one or each eye \\
    	\hline
     	Virtual Reality (VR) & A digital world where the user gets immerged. This cannot be seen in the real world
			\\
    	\hline
	Virtual Environment (VE) & Replicates the real world and the users presence and interactions inside it.  \\
     	\hline
     	360 video & A video feed that can be seen from all angles, not only a specific camera angle. \\
     	\hline

\end{tabular}
