\section{Problem}

\subsection{Purpose}
\label{intro:purpose}
As companies might shift their focus towards creating experiences in virtual reality, their design process have to adapt according to the features of this new medium. As a part of that, this thesis will evaluate the opportunity of a VR tool for manipulating 3D environments while immersed in VR into their new VR design process. The purpose of this is to elevate the experience when creating and using the end product, as designing 3D environments and experiences is highly dependent on the actual perspectives and positions of the end user. When designing these environments on a screen some aspects of the design becomes hard to predict when seen on a regular screen.

\subsection{Objective}
By allowing the designer and developer to manipulate the environment of their application when immersed might solve issues explained in section \ref{intro:purpose} and create a more intuitive user experience during creation and manipulation of objects in the application. By creating a tool that can allows this type of workflow, the goal is for the user to make instant tweaks to the world without having to leave it. An evaluation of this tool will also be conducted, if it can be used for rapid prototyping and bridge the knowledge-gap between team-members by letting non-developers edit the prototype and explain their ideas.


\begin{itemize}
	\item Research possibilities and drawbacks of using VR as the medium for creation and manipulation.
	\item Study how VR projects are carried out today through interview, then identify opportunities where immersion is beneficial
	\item Design VR interface prototypes that are user-friendly and task specific from the results of the research and study
	\item Build functional prototype based on the final design
	\item Prototypes will be evaluated with user tests
\end{itemize}

\subsection{Limitations}
This thesis will be based on the current technology and software on the market. The concepts and prototypes created will be restricted to the possibilities of what can be implemented into a working system today. The scope of what will be investigated is restricted to the field of VR experience and experience design along with its implementations. The targeted user-base will be limited to working professionals within the field of tech and experience design. The hardware HMD selection will be limited to the available hardware at the Stockholm office of North Kingdom.

\subsection{Terminology}
\begin{tabular}{ | p{5.5cm} | p{6.5cm} |  }
	\hline
     	\multicolumn{2}{|c|}{Common terms and their explanation} \\
     	\hline
	Field of View (FOV) & What a user can see without turning head or body \\
    	\hline
     	Head-Mounted Display (HMD) & A display device worn on the users head with a small screen in front of one or each eye \\
    	\hline
     	Virtual Reality (VR) & A digital world where the user gets immersed. This cannot be seen in the real world
			\\
    	\hline
	Virtual Environment (VE) & Replicates the real world and the users presence and interactions inside it.  \\
     	\hline
     	360\degree video & A video feed that can be seen from all angles, not only a specific camera angle. \\
     	\hline
Degrees Of Freedom (DOF) & The number of parameters in which an object or a system may vary independently. \\
\hline
\end{tabular}
