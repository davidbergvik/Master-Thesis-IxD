\section{Interviewee name: David Ljunghill}
Date: 2017-03-14
Dictionary
Word: Explanation
\subsection{Questions}
\begin{enumerate}
\item Where do you work?
North Kingdom
\item What's your title?
UX Designer
\item How long have you been working there?
8 months here (total of 5 years as professional UX)
\item How many VR projects have you been involved in?
Starting my first one now, been reading and studying how to do UX-work for VR for a while.
\item What are the biggest problems you meet in your daily work?
\begin{itemize}
  \item How to get into the process of creating VR
  \item Been working on a screen for so long, used to that
  \item know how to develop an application by myself and create prototypes and proof of concept that way.
  \item Feel left out in the cold as i have to depend on others to create environment in VR for me
  \item I need a tool that does not require an education to use (There is no time to becomes an expert in unity just to create prototypes)
\end{itemize}
\item What kind of design process do you append to VR projects?
For "normal" screen project I:
\begin{itemize}
  \item Sketch using pen and paper/tablet
  \item Create high def using Sketch
\end{itemize}
\item Are there flaws with this design process in VR projects?
\begin{itemize}
\item I lack a tool like Sketch for
\begin{itemize}
  \item creating fast UI elements and prototypes
  \item add animations and transitions between states (drag and drop)
  \item No coding
  \item (premade) Animations are important, simple parameters that can be changed in editor
  \begin{itemize}
    \item Duration
    \item Speed
    \item Acceleration
  \end{itemize}
  \item Placing and arranging objects in the virtual space, get a feel for proportions and depth
  \item getting from a sketch on paper into a VR environment in order to communicate to the rest of the team
  \item Fast creation, it cannot take time
  \item Being able to think out loud through the tool, without the tool distracting me from my creative "zone"
  \item Acceleration
\end{itemize}
\item Limitations are good
\item Not having to create light sources and spaces (when its not priority)
\item Get of the ground running
\item Presets of scenes with different scales. Not having to create this
\item It's hard to get started with creating in VR, where do I start?
\item Connecting a mobile VR headset to the workstation/computer
\item No way to preview what you've done on the computer, no sync between mobile phone/VR headset and what I'm doing.

\end{itemize}
\item Are there any time-consuming processes in the pipeline?

\item How would you like to work in the future?
\begin{itemize}
\item Working with a phone-based headset is preferable
\item Better to go from bad quality than a super high def
\item Everyone have a phone, there is only one Vive in the office (in another room)
\item Right now I need a tool for getting the ordinary screen UI elements (buttons, labels, containers/sheets etc) into a VR environment to try out concepts with them.
\item Hopefully I would not have to sit in Unity
\end{itemize}
\item How does the communication within the project core team work?

\item How are prototypes utilized in your VR projects?
Previz, see other interviews.
\item How do you describe a VR concept or world to your teammates? (Which tools do you use?)
\begin{itemize}


\item Sketching a scene or interface depending on the interface. (kind of a wireframe)
\item It works quite well as a communication tool.
\item The more apparent it is that I've not spent any time on this idea, the easier it is for my listener to engage and give suggestions and changes.
\item Sketching while explaining the idea is great, then people can make suggestions and I can incorporate them as I'm sketching
\item I don't wanna spend 4 hours creating something that I already know.
\item I want the possibility to just drag and drop objects to make a UI in VR, then when I'm satisfied with the concept I'll show it to my team.

\end{itemize}
\item Have you tried applications where you manipulate the VR world?
Only Tilt Brush
\item What do you think about developing/prototyping for VR while emerged?
\begin{itemize}


\item I don't think that we will work only from within a VR world,
\item Will possibly hurt eyes to sit in VR a whole day.
\item The computer is used for a lot of research and other aspects, so it will still be a major tool for our day to day work
\item But it can act as a tool for preview, editing and testing concepts.
\end{itemize}
\item What do you think of this kind of tool within your current pipeline?
\begin{itemize}
\item Good for exploration as well as a testing ground for concepts
\item Place and change details (tweak) things in VE. Not having to change between VR and computer for that.
\item Manipulate objects are prio ONE.
\item Having properties available for objects, changing parameters inside VR.
\item Repeat animations while tweaking timings and distances.

\end{itemize}
\end{enumerate}
Other notes

Name for tool: VRify!!

Create animation as a gameObject (a line that is visible when the object is selected)

Not having a HUD, it's weird for the brain in VR

Sitting/standing at a desk is preferable, compared to walking and turning.

Create a design for lower fidelity (one pointer) and complexify for hifi hardware like Vive.

You want to be as relaxed and still as possible.

Focus on creating a single user tool, then move towards a workshop/brainstorm tool for collaboration.
