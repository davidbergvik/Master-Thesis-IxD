
\chapter{Discussion}
\label{discussion}
*** Create intro ***
\section{Multiple Tools}
After the interviews it became clear that two different solutions should be implemented, in best case. One of the biggest advantages of having two different tools is that a a tool for a low-level medium (created for a less powerful device) this tool could be used by several people in parallel with the much more convenient and portable Google Daydream\footnote{https://vr.google.com/daydream/}.Unfortunately this was tweaked however as the VR editors that exists only support hi-fidelity HMD's (HTC Vive\footnote{https://www.vive.com/eu/} and Oculus Rift\footnote{https://www.oculus.com/rift/}) and would therefore require a different implementation approach.
\section{Prototype Testing}
\subsection{Lo-fi}
The initial user tests of the lo-fi concepts where perhaps the most game-changing in terms of major changes to the prototype. Since the tests consisted of the users trying to interact with the tool and the environment, they are not informed on what kinds of interactions that are available. This gives a better view om the logic of what the user expects from the environment. For example: The initial concepts that involved movements on the trackpad of the controller for selection within a pie-menu (Section \ref{theory:interface}). With knowledge of the interaction patterns, the concept seemed really thought-out and logical, as well as it included ease of use. When testing these features however, none of the users actually tried to use this technique without hints from the test leader. Instead they tried to point to items on the menu with the ray-cast (Section \ref{theory:toolsandtech:raycast}) until they were told that nothing happens. This result strengthened the idea of pointing instead of using the touch pad, and when users were asked why the came to that conclusion, the answer where because they pointed and clicked on the menu to open it. At the time of testing there were no cues to actually introduce the possible interaction with the touchpad and because of that, this approach was still carried into the hi-fi prototype as a concept as well.

Another insight unexpectedly appeared during the user testing of the lo-fi prototypes was the importance of the scene hierarchy. In this, all objects of the scene are listed as subobjects of each other. Initially the design did not include this, as to eliminate all complex features that could distract the user from their primary objective, to create. Without this however, the sequence of creating a new object gave the user no relation between the new object and the rest of the environment. The usage of the hierarchy also enables users to interact with and manipulate occluded objects.
\subsection{Hi-fi}
After conducting tests on the lo-fi prototype, the selector interface was prioritized higher than a controller-based interface for access to object manipulation tools. This was due to its direct connection with the object and that it was easier to understand when testing in the real world. When the hi-fi prototype where developed and initially tested, this changed (again). When the user selects an object that is far away, the selector next to it appeared so small that it became very hard to read and interact with it. Perceptive scaling were added to the interface to avoid this, which caused the interface to occlude other objects adjacent to the selected object (more about occlusion in section \ref{theory:interactionissues:occlusion}). These problems shed some new light on the previously ignored solution, a controller-based interface using the touchpad. The interactions with this interface were chosen because they can be achieved using one controller , which supports both middle- and high-level devices (more information about devices in section \ref{theory:HMD}), and its accessibility. The apparent issue with this approach that was discovered in lo-fi tests were addressed by changing the color of the bounding box when the menu is activated. The idea is to connect the touchpad interactions with a specific color, that separates it from the primary interactions done with the raycast.

\subsection{Tiltbrush as a sketching tool}
