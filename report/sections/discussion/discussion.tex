
\chapter{Discussion}
\section{Lo-fi Prototype}
\subsection{Testing}
The inital user tests of the lo-fi concepts where perhaps the most gamechanging in terms of major changes to the prototype. Since the tests consisted of the users trying to interact with the tool and the environment, they are not informed on what kinds of interactions that are avaliable. This gives a better view om the logic of what the user expects from the environment. For example: The initial concepts that involved movements on the trackpad of the controller for selection within a pie-menu (Section \ref{theory:interface}). With knowledge of the interaction patterns, the concept felt really thoughout and logical, aswell as it included ease of use. When testing these features however, none of the users actually tried to use this technique without hints from the test leader. Instead they tried to point to items on the menu with the ray-cast (Section \ref{theory:toolsandtech:raycast}) until they were told that nothing happens. This result strengthend the idea of pointing instead of using the touch pad, and when users were asked why the came to that conclusion, the answer where because they pointed and clicked on the menu to open it. At the time of testing there were no cues to actually introduce the possible interaction with the touchpad and because of that, this approach was still carried into the hi-fi prototype as a concept aswell.

Another insight unexpectedly appeared during the user testing of the lo-fi prototypes was the importance of the scene hierarachy. In this, all objects of the scene are listed as subobjects of eachother. Initially the design did not include this, as to eliminate all complex features that could distract the user from their primary objective, to create. Without this however, the sequence of creating a new object gave the user no relation between the new object and the rest of the environment. The usage of the hierarchy also enables users to interact with and manipulate occluded objects.

\subsection{Tiltbrush as a sketching tool}

\section{Multiple Tools}
After the interviews it became clear that two different solutions should be implemented, in best case. This did however not happen as the idea hit roadblock upon roadblock. One of the biggest advantages of having two different tools is that a a tool for a low-level medium (created for a less powerful device) this tool could be used by several people in parallel with the much more convinient and portable Google Daydream\footnote{https://vr.google.com/daydream/}. Unfortunatly this was scraped as the VR editors that exists only support hi-fidelity HMD's (HTC Vive\footnote{https://www.vive.com/eu/} and Oculus Rift\footnote{https://www.oculus.com/rift/}) and would therefore require a different implementation approach.
