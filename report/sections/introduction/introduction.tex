
% !TEX encoding = UTF-8 Unicode
\chapter*{Introduction}
Virtual reality (VR) is a technology that's been around for decades and has in the last few years into a whole new set of entertainment, and keeps getting more traction each year
\cite{VR:mazuryk1996virtual}
VR is a concept in which the user is placed in a virtual reality using a headset or other hardware. As this is the type of VR we see today, exercised technology Already in the 1800s with the panoramic paintings that stretches 360 degrees around the spectator.

In recent years, the variety of VR headsets has exploded and technology became during 2016 a billion \$ -industry , which is expected to continue increasing in the coming years.
\cite{VR_stats:statista}
This increase created a tremendous increase of available application for VR and there are as of 2016 hundreds of companies working with developing experiences for this new medium.

THESE ARE THE DIFFERENT SECTIONS THAT THE INTRO WILL CONSIST OF
\begin{quote}
\section{Introduction to the introduction} The first step will be a short version of the three moves, often in as little as three paragraphs, ending with some sort of transition to the next section where the full context will be provided.
\section{Context}: Here the writer can give the full context in a way that flows from what has been said in the opening. The extent of the context given here will depend on what follows the introduction; if there will be a full lit review or a full context chapter to come, the detail provided here will, of course, be less extensive. If, on the other hand, the next step after the introduction will be a discussion of method, the work of contextualizing will have to be completed in its entirely here.
\section{Restatement of the problem}: With this more fulsome treatment of context in mind, the reader is ready to hear a restatement of the problem and significance; this statement will echo what was said in the opening, but will have much more resonance for the reader who now has a deeper understanding of the research context.
\section{Restatement of the response}: Similarly, the response can be restated in more meaningful detail for the reader who now has a better understanding of the problem.
\section{Roadmap}: Brief indication of how the thesis will proceed.
\end{quote}
