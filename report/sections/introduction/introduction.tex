
% !TEX encoding = UTF-8 Unicode
\chapter*{Introduction}
Creating immersive 3D environments for virtual reality (VR) experiences presents designers and developers with new challenges that did not exist for creating experiences on a screen. As the medium for the experience differs from the medium on which it is created, designers and developers have to adapt new design processes and pipeline for creating and manipulating 3D environments in VR. These professionals needs a new toolset to effectivly work with this new medium as it's emerging into an entire industry.\footnote{https://www2.deloitte.com/global/en/pages/technology-media-and-telecommunications/articles/tmt-pred16-media-virtual-reality-billion-dollar-niche.html} There have been a lot of research within similar fields of 3D manipulation with a Head Mounted Display (HMD),  \cite{relatedwork:kijimaand1997transition} \cite{relatedwork:bowman1996conceptual} \cite{relatedwork:stoakley1995virtual} \cite{relatedwork:mine1995isaac}, but recent technological strides demands new efforts to translate earlier knowledge toward this new type of profession and what tools are needed.

%\section{3D Graphics}
Since the world was introduced to computers more then five decades ago, a great number of new professions and companies have emerged and the technology has created more jobs then before. \footnote{https://www.theguardian.com/business/2015/aug/17/technology-created-more-jobs-than-destroyed-140-years-data-census} One of the applications that harnesses the power of computer systems are computer graphics, a term coined by William Fetter in 1960. \cite{3D_history:graphics_2017} In the 1990's the graphics took the leap into 3D modelling for the cosumer market, and today a user can create complext 3D environments on their personal laptops. 3D designers creates and manipulates 3D environments using powerful computers, high-resolution screens and a number of input tools like pens, pads and keyboards.


%\section{Virtual Reality}
Virtual reality (VR) is a technology that's been around for decades and has in the last few years into a whole new set of entertainment, and keeps getting more traction each year
\cite{VR:mazuryk1996virtual}
VR is a concept in which the user is placed in a virtual reality using a headset or other hardware. As this is the type of VR we see today, exercised technology Already in the 1800s with the panoramic paintings that stretches 360 degrees around the spectator. In recent years, the variety of VR headsets has exploded and technology became during 2016 a billion \$ -industry , which is expected to continue increasing in the coming years.
\cite{VR_stats:statista}
This increase created a tremendous increase of available application for VR and there are as of 2016 hundreds of companies working with developing experiences for this new medium.


%\section{Objective}
The objective of this thesis is to analyze the current process of creating 3D experiences in VR and possible challenges that arise. From the analyzis the use of virtual reality and HMDs when creating an immersive 3D environment will be evaluated. Furthermore the VR tool will be evaluated tested to bridge the knowlegde gap between the core project team during concept and prototyping phases.

%THESE ARE THE DIFFERENT SECTIONS THAT THE INTRO WILL CONSIST OF
%\begin{quote}
%\section{Introduction to the introduction} The first step will be a short version of the three moves, often in as little as three paragraphs, ending with some sort of transition to the next section where the full context will be provided.
%\section{Context}: Here the writer can give the full context in a way that flows from what has been said in the opening. The extent of the context given here will depend on what follows the introduction; if there will be a full lit review or a full context chapter to come, the detail provided here will, of course, be less extensive. If, on the other hand, the next step after the introduction will be a discussion of method, the work of contextualizing will have to be completed in its entirely here.
%\section{Restatement of the problem}: With this more fulsome treatment of context in mind, the reader is ready to hear a restatement of the problem and significance; this statement will echo what was said in the opening, but will have much more resonance for the reader who now has a deeper understanding of the research context.
%\section{Restatement of the response}: Similarly, the response can be restated in more meaningful detail for the reader who now has a better understanding of the problem.
%\section{Roadmap}: Brief indication of how the thesis will proceed.
%\end{quote}
