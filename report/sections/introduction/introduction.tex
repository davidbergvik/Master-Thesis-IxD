
% !TEX encoding = UTF-8 Unicode
\chapter{Introduction}
Creating immersive 3D environments for virtual reality (VR) experiences presents designers and developers with new opportunities and challenges that does not appear when creating experiences on a screen. As the medium for the experience (Head Mounted Display) differs from the medium on which it is created (desktop PC), designers and developers have to adapt new design processes and pipeline for creating and manipulating 3D environments in VR. These professionals needs a new toolset to effectively work with this new medium \footnote{http://www.gamasutra.com/view/news/288508/} as it's emerging into an entire industry\footnote{https://www2.deloitte.com/global/en/pages/technology-media-and-telecommunications/articles/tmt-pred16-media-virtual-reality-billion-dollar-niche.html}. There have been a lot of research within similar fields of 3D manipulation with a Head Mounted Display (HMD)  \cite{relatedwork:kijimaand1997transition} \cite{relatedwork:bowman1996conceptual} \cite{relatedwork:stoakley1995virtual} \cite{relatedwork:mine1995isaac}, but recent technological strides demands new efforts to translate earlier knowledge toward new application of these professions and what tools are needed\footnote{http://www.marxentlabs.com/job/virtual-reality-jobs/}.

%\section{3D Graphics}
Since the world was introduced to computers more than five decades ago, a great number of new professions and companies have emerged as this new technology creates more jobs for society\footnote{https://www.theguardian.com/business/2015/aug/17/technology-created-more-jobs-than-destroyed-140-years-data-census}. One of the applications that harnesses the power of computer systems are computer graphics, a term coined by William Fetter in 1960\cite{3D_history:graphics_2017}. In the 1990's the graphics took the leap into 3D modelling for the consumer market, and with the increased processing power in todays computers, a user can create complex 3D environments on their personal laptops. 3D designers creates and manipulates 3D environments using powerful computers, high-resolution screens and a number of input tools like pens, pads and keyboards.


%\section{Virtual Reality}
VR is a technology that's been around for decades and has in the last few years developed into a whole new set of entertainment, and keeps getting more traction each year\cite{VR:mazuryk1996virtual}. VR is a concept in which the user is placed in a virtual reality using a headset or other hardware. This is the type of VR we see today, but the concept was used  as far back as the 1800s, with the panoramic paintings that stretches 360 degrees around the spectator. In recent years, the variety of VR headsets has exploded and technology became during 2016 a billion-dollar-industry and is expected to continue increasing in the coming years\cite{VR_stats:statista}.
This increase created a tremendous increase of available application for VR\footnote{https://www3.oculus.com/en-us/blog/gear-vr-ecosystem-expands-to-include-facebook-360-photos-over-250-apps-and-new-video-content/} and there are as of 2016 hundreds of companies working with developing experiences for this new medium\footnote{http://venturebeat.com/2017/03/08/vr-companies-grew-40-percent-in-2016/}.

%\section{pitch!}
The process of creating 3D experiences in VR comes with challenges and issues not previously seen in this field. Is there a way of creating within this new medium and in that case, what could that solution look like? Is this preferable to the current way of working? Lets find out!

\section{North Kingdom}
This thesis will be carried out together with the company North Kingdom. The company defines itself as an "experience design company", founded in Skellefte\aa and has been a public company since 2003. Today, the offices of North Kingdom are in Skellefte\aa, Stockholm and Los Angeles. \footnote{http://www.northkingdom.com/} Their vision is
\begin{quote}
We Believe That new value can be created wherever people, business, and technology collide. We help our clients harness That value through the creation of experiences, products, and services That play a meaningful role in people's lives. Through human-centered design, we make the complex simple and relatable, no matter what medium or platform\footnote{http://www.northkingdom.com/}.
\end{quote}
This thesis will run in parallel with North Kingdom's current VR projects. Their design process and pipeline will act as the scope for this thesis.

\section{Experience Design}
Definitions of the term "Experience Design" is widely spread out throughout design communities, pinpointing different parts of the design spectrum. The main concept however is explained by Marzano as a process of focusing on the quality of the user experience and shifting the focus from expanding sets of features towards solutions that is tailored to the current cultural state of society\cite{experience_design:marzano2003new}. Furthermore Marzano explains how companies like Philips Design focuses on the complete experience that a user has with their brand. This extends from first impression, through the usage lifecycle and to the overall experience. Marzano presents a design process created by Philips Design described as "The Experience Design Lifecycle", which consist of three main activity phases; Envisionment, Experience Concepts and Experience Centre. These concepts will be explained further in the following section.

\textbf{Envisionment:}
This stage is where technical developers and designers collaborate to understand the possibilities and affects of different design elements, isolated from their intended context. This is carried out using rapid prototyping (more details in section \ref{method:prototype}) and is explored with isolated elements and as a collaboration between different elements.

\textbf{Experience Concepts:}
Pinpointing cultural trends and human needs is essential in this stage when inital concept ideas are being explored. The core aspects defined as the foundation of this phase are People, Space and Enablers over time. This provides a context which is used to create a  unique and tailored experience.
\begin{itemize}
  \item The people aspect focuses on the individuality of people and the uniqueness of their individual experience. This re-aligns the focus of the design, from technological capabilities to experiences of the users.
  \item Space symbolizes the context of which the concept or product is perceived and used. The focus of this aspect is to utilize and account for the space and how it affects the user.
  \item Enablers are the technological advancements as well as design enablers that connects the system to the user. The key to success in this aspect lays in the combination of these tools in reference to the users needs and capabilities.
  \item All of these three have to be adapted to the presence of time and how that affects the attention of the user. The result from this stage are experiental prototypes that embodies the core features and elements of the concepts, which is used as a demonstration of the ideas as well as a subject for testing and evaluation.
\end{itemize}

\textbf{Experience Centre:}
 At this stage the concept is brought into the wild, the real world, through the creation of prototypes that can withstand proper usage and placement in its targeted environment. This provides insights of affects, usage and acceptance on real users. The cost and effort of creating these prototypes are much higher than the experiental prototypes created in the previous step in order to give accurate responses from users. This phase also implies that a 'parent-child' approach should be considered if a new technology is being introduced to the environment. This means that the new emerging technology is supported by an established technology as a foundation.
