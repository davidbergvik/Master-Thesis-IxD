\section{Iterative Prototyping}
\label{method:prototype}
In order to test the concepts from problems that were discovered in the interview phase, interfaces were designed and developed by utilizing iterative prototyping. Prototyping is a paradigm that is embodied in the \textquote{throw one away philosophy} of system development\cite{proto:Gomaa1981}. Iterative prototyping keeps refining a design by creating partially working prototypes and user-testing them at a fast pace to get early and regular observation of the system and interface behavior\cite{proto:hartson2012ux}.

One of the main benefits with rapid prototyping is that the approach might reveal misunderstandings between designers and developers and the users, and are in line with the findings from empirical research about designers and how they work\cite{proto:tripp1990rapid}. These missunderstandings can originate from differences in backgrounds and/or experience as Gomaa and Scott concludes in a study about using prototypes throughout a software development process.\cite{proto:Lichter1993} The authors goes on to state that creating a prototype in order to evaluate a product specification is both effective and less expencive than shipping straight to production.

This phase was divided into short sprints and carried out according to the design, test, evaluate principles from iterative prototyping.\cite{proto:hartson2012ux} The focus and softwares used for these steps can be found in Table \ref{table:sprints}. The created prototypes along with the results from testing are found at \ref{result:prototypes}

\subsection{Lo-fi}
\label{method:prototype:lofi}
The first step when prototyping is often to sketch it up on a piece of paper. This is proven to be one of the best methods this early in the process as the rough estetics provides a platform where it's easy to create, easy to discard ideas and are more encouraging of critisism from test subjects. \cite{proto:boling1997holistic} After inital prototypes where made, the concepts where evaluated by drawing them in Tilt Brush. Doing this might remove some ideas just because what work on paper might not work in 3D and VR. There has been instances where VR games and projects have used the previous mentioned application Tilt Brush (see Section \ref{relatedwork:tiltbrush})for prototyping purposes. The company Dream On VR \footnote{https://dreamonvr.com} is an example as used the application to prototype level and map designs \footnote{https://uploadvr.com/til-brush-game-levels-prototype/} for their VR superhero game.

The testing of these prototypes for VR purposes comes with new challenges. For testing a design created for a screen, hand-drawn sketches of the different views and scenarios from the application are displayed one at a time to the user. When the user "interacts" with something on the sketch, a new sketch is displayed\cite{proto:boling1997holistic}. This is not possible on paper when testing for the purpose of VR.

The prototypes were created in large numbers of different concepts as sketches on paper. These concepts had different approaches according to the results of interviews (Section \ref{result:interviews}). All concepts were then evaluated based on the theoretical framework (Section \ref{theory}) and through consultations with UX professionals working at North Kingdom. They were then tested in a 'real-world' scenario where the VE was substituted with the real world using an approach called 'Wizard of Oz', where a 'wizard' (test leader) simulates actions and behaviour of a system\cite{proto:landauer1987psychology}. With this approach a system can be tested without full implementation. The test subjects was given a real 3 DOM controller and were instructed to perform certain task with objects in the real world. The UI of these tests where created on post-its (smaller UI components) and sheets of paper (larger UI). As the users interactions was not tracked, every move, point and interaction were explained orally as the test leader acted as the 'wizard' and manipulated objects and UI depending on user interactions.
\subsection{Hi-Fi}
\label{method:prototype:hifi}
When a basic understanding of the problem and the general concept had been developed, hi-fidelity prototypes were developed. The key difference in this step compared to the previous lo-fidelity prototype is covering more dimensions of the desired product. In their study about the validity of hi-fi prototypes, Virzi et. al concludes that this is a nessecity in order to discover problems with a design if the lo-fi prototype lacks certain dimensions\cite{proto:virzi1996usability}. For this concept, the missing dimension is mainly the properties of a virtual reality that is lacking in the real world. Properties like teleportation, changing scale of 3D objects, remote object placing etc.
These prototypes were created as interfaces in  Sketch\footnote{https://www.sketchapp.com/} and Adobe Illustrator\footnote{http://www.adobe.com/se/products/illustrator.html} which are softwares used for creating digital design elements and interfaces in vector format. All interfaces were exported as UI elements into Unity using the plugin Fetch\footnote{http://fetchui.com/}. This allowed for an interactive test-suite that was created for HCT Vive in Unity for user testing purposes.
This was achieved using SteamVR\footnote{https://support.steampowered.com/kb\_article.php?ref=2001-UXCM-4439} and a template VR scene created by Ray Wenderlich\footnote{https://raywenderlich.com/149239/htc-vive-tutorial-unity}. The scope of these test included understanding of the basic functionality, proper testing of relative sizes and proportions, user flows and overall user experience.

\begin{table}[]
  \centering
  \caption{Sprint schedule for prototype phase}
  \label{table:sprints}
  \begin{tabular}{|l|l|l| p{5cm}|}
     \hline
    \textbf{Duration} & \textbf{Fidelity} & \textbf{Tools used} & \textbf{Focus} \\\hline
    1 sprint                         & Lo-fi prototype  & Pen, paper, post-its    & Experiment with relationship of different components                               \\\hline
    1 sprint                        & Hi-fi prototype & Sketch, Sketch-to-Unity & Investigate the design of isolated parts and components.                           \\\hline
    2 sprints                 & Interactive   & Unity, Unity VR Editor  & Evaluate different types of interactions between the interface and the environment  \\\hline
  \end{tabular}
\end{table}

\subsubsection{User testing and Evaluation}
The prototype was evaluated more through user tests and follow-up interviews. The purpose of this evaluation is to asses the usability and validity of this VR editor concept. The test was based on a series of tasks that the subjects carried out. Each task used at least one of the modules that where developed based on the inital interviews.
The test subjects where developers and art directors and interactiondesigners at North Kingdom with different levels of experience of working with Unity \footnote{https://unity3d.com/unity}. This approach allowed comparisions between experienced and no experience users and test if this concept could work as a mediator-tool in cross-functional teams. After each test an interview was conducted in order to get qualitive data about the perfomance and how this concept works compared to the process that is used today. Th prototype were limited the prototype to only handle the functionality for scenarios including the dimensions that were missing in previous tests. Some of these limitations could be handled by the test leader using the Wizard of Oz approach\cite{proto:landauer1987psychology}, by changing the VE from the computer that is running the test suite, as the test subject is immerged and unaware of actions that the test leader makes during runtime.
The tasks that were included in the tests are:
\begin{itemize}
  \item Move around in the VE
  \item Select an object
  \item Move objects to a certain position along their reference planes
  \item Move object along their third axis (move the reference plane). Stacking objects on top of each other
  \item Delete an object
  \item Create a new object that is not already in the scene
  \item Duplicate an object
  \item Add a property to an object
\end{itemize}
