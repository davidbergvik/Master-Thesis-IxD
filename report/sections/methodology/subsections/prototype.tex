\subsection{Iterative prototyping}
Prototyping is a paradigm that is embodied in the “plan to throw one
away” philosophy of system development [4].
In order to test the concepts from problems that were discovered in the interview phase, interface designs were designed and developed by utilizing iterative prototyping within the design process. Iterative prototyping keeps refining a design by creating partially workingh prototypes and user-testing them at a fast pace to get early and regular observation of the system and interface behavior. One of the main benefits with rapid prototyping is that the approach reveals misunderstandings between
designers and developers and the  users. These missunderstandings can originate from
 differences in backgrounds and/or experience as Gomaa and Scott conludes in a study about using prototypes throughout a software development process. The autors goes on to state that creating a prototype in order to evaluate a product specification is both effective and less expencive than shipping strait to production.
[Prototyping as
a tool in the specifications of user requirements]
