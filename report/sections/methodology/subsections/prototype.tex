\section{Iterative Prototyping}
In order to test the concepts from problems that were discovered in the interview phase, interfaces were designed and developed by utilizing iterative prototyping. Prototyping is a paradigm that is embodied in the plan to throw one away philosophy of system development. \cite{proto:Gomaa1981} Iterative prototyping keeps refining a design by creating partially working prototypes and user-testing them at a fast pace to get early and regular observation of the system and interface behavior.\cite{proto:hartson2012ux}

One of the main benefits with rapid prototyping is that the approach might reveal misunderstandings between designers and developers and the users, and are in line with the findings from empirical research about designers and how they work\cite{proto:tripp1990rapid}. These missunderstandings can originate from differences in backgrounds and/or experience as Gomaa and Scott concludes in a study about using prototypes throughout a software development process.\cite{proto:Lichter1993} The authors goes on to state that creating a prototype in order to evaluate a product specification is both effective and less expencive than shipping straight to production.

\subsection{Prototype Fidelity}
\subsubsection{Lo-fi}
\label{method:prototype:lofi}
The first step when creating a prototype is often to sketch it up on a piece of paper. This is proven to be one of the best methods this early in the process as the rough estetics provides a platform where it's easy to create, easy to discard ideas and are more encouraging of critisism from test subjects. \cite{proto:boling1997holistic} After inital prototypes where made, the concepts where evaluated by drawing them in Tilt Brush. Doing this might remove some ideas just because what work on paper might not work in 3D and VR. There has been instances where VR games and projects have used the previous mentioned application Tilt Brush (see Section \ref{relatedwork:tiltbrush})for prototyping purposes. The company Dream On VR \footnote{https://dreamonvr.com} is an example as used the application to prototype level and map designs \footnote{https://uploadvr.com/til-brush-game-levels-prototype/} for their VR superhero game.

The testing of these prototypes comes with new challenges. For testing a design that will be displayed on a screen, hand-drawn sketches of the different views and parts of the application are displayed one at a time to the user. when the user "interacts" with something on the sketch, a new sketch is displayed. This is not possible on paper because of the 3D aspect and the fact that the environment and interface might not attached like a 2D application.
\subsubsection{Hi-Fi}
\label{method:prototype:hifi}
\subsubsection{Interactive}
\label{method:prototype:interactive}
This phase was divided into short sprints and carried out according to the design, test, evaluate principles from iterative prototyping.\cite{proto:hartson2012ux} The focus and softwares used for these steps can be found in Table \ref{table:sprints}.

\begin{table}[]
  \centering
  \caption{Sprint schedule for prototype phase}
  \label{table:sprints}
  \begin{tabular}{|l|l|l| p{5cm}|}
     \hline
    \textbf{Duration} & \textbf{Fidelity} & \textbf{Software used} & \textbf{Focus} \\\hline
    1 sprint                         & Lo-fi prototype  & Pen, paper, post-its    & Experiment with relationship of different components                               \\\hline
    1 sprint                        & Hi-fi prototype & Sketch, Sketch-to-Unity & Investigate the design of isolated parts and components.                           \\\hline
    2 sprints                 & Interactive   & Unity, Unity VR Editor  & Evaluate different types of interactions between the interface and the environment  \\\hline
  \end{tabular}
\end{table}
