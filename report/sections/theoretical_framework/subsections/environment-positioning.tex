\section{Environment Positioning}
Being able to position yourself within a virtual environment is fundamental in order to interact with it and is important to investigate.\cite{tools:warren2014perception} Even if the physical space around the user is confined (see section \ref{theory:bodyLimits} for more information), the virtual space can be infinite. In order to harness the power of an immense virtual environment, some kind of positioning tool or traveling technique can be implemented. This will allow the user to interact with distant object through direct manipulation, get a different perspective of the virtual world and more as explained by Robinett and Holloway. \cite{positioning:Robinett1992} The authors present in the study the three actions that can be implemented into a virtual environment. These three are fundamental for positioning:
\begin{itemize}
  \item Translate - Move object, teleport/transport the user
  \item Rotate - Grab and turn an object, tilt the entire world
  \item Scale - Scale an object, expand/shrink the world/user
\end{itemize}

A positioning system can be created by selecting and combining tools and techniques in Section \ref{theory:toolsandtech}.
