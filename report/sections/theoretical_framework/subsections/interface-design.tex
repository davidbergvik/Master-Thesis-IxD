\section{Interface Design}
Interfaces in a virtual environment comes with a new set of challenges when compared to a traditional interface designed for a screen. If these challenges are neglected the virtual experience for the user might be troublesome.

When designing an interface it's important to evaluate the speed of selection. Fitts' law is a fundamental and proven way to evaluate pointing to real-world objects by measuring the distant to the object and its size.\cite{interface:Fitts1954}

This is however based on a real-world scenario, which does not translate into a virtual environment, were the user need a tool to interact with objects. Despite these differences, Fitts' law can be applied to pointing in a virtual environment using the following formula  \cite{interface:card1978evaluation}
\begin{equation}
ID = log(2) * ( 2* D / W )
\end{equation}

where ID is the index of difficulty, D is the distance from startingpoint to the middle of the correct target. W is the width of the target (caluculated  on the axis where the pointer will travel).

\subsection {Interface Design in VR}
