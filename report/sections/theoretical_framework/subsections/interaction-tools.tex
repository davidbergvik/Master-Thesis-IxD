\section{Interaction Tools in VR}
There is a lot of parameters in the virtual world to take into account when deciding on a selection tool and technique. Some of the tools and techniques that exists for VE will be explained.

\subsection{Raycast techniques}
Multiple studies have concluded that in an environment with a sparse selection of objects with a volume that is not too small, using a raycast is most really fast and reliable. [insert raycast (pointer) references here] When some of these parameters change however, the raycast tool with a "laser-pointer" technique experiences more issues. In an environment that contains alot of objects in a small space, the error rate rises. This factor is multiplied when movement is added to the object (typical for games).

Accordning to %Dense and Dynamic 3D Selection for Game-Based Virtual Environments
 there are better ways to perform object selection in a more complex and dynamic environment by tweaking this concept. By using techniques that are designed for dynamic and cluttered environment the speed and error rate can be reduced. Two of these techniques are 'zoom' and 'expand'. On first selection the surrounding area of the selected object is enhanced to simplify the selection.

A big problem with pointing is trembling of the hand and twitches that occour when user tries to select an option. This has been given the name "Heisenberg effect" and is the cause of new interaction issues:

\begin{itemize}
\item user dissatisfaction due to increased error rates,
\item discomfort due to the duration of corrective movements, which in the absence of physical support require an additional physical effort, and
\item unconfidence on which object will be selected after triggering the confirmation
\end{itemize}
[Improving 3D Selection in VEs through Expanding Targets and Forced Disocclusion]
