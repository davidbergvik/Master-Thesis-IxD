\section{Interaction Tools in VR}
\label{theory:toolsandtech}
There is a lot of parameters in a VR to take into account when deciding on a selection tool and technique, which is explained by the great number of studies that have been conducted in this field of VR \cite{tools:poupyrev1996go,tools:mine1997moving,tools:Cutler1997,tools:bowman2001introduction}. Some of the tools and techniques that exists for VE will be explained.

\subsection{Direct Manipulation}
One of the best and most intuitive ways of interacting in VR are using direct manipulation.\cite{tools:jacoby1994gestural} This technique mimics the way that humans interact and manipulate objects in the real world; Grabbing with our hands. There have been studies with variations of this technique which have proved that it performes great as an intuiative tool for interacting in VR.\cite{tools:Buchmann2004,tools:Cutler1997} These studies also conclude that direct manipulation is preferable when the target users are novices at using VR. Direct manipulation can also be used for traveling within the VE by using Scale-World grab, which was created by Mine.\cite{tools:mine1997moving} This allows the user to move around by grabing inside the VE and moving hands to the oposite direction of where the desired destination is. For example: If the user wants to go forward then they'll extend their arms formward, grab, then pull thier hands towards their body. Furthermore Mine explains how the Virtual Hand (direct manipulation) can be combined with a raycast (from section \ref{theory:toolsandtech:raycast}) to provide delicate object manipulation and also reach occluded objects. This is explained as a technique shift between the two that occurs when the hand/controller appears/disappears from the users field of view.

\subsection{Raycast techniques}
\label{theory:toolsandtech:raycast}
Studies have concluded that in an environment with a sparse selection of objects with a volume that is not too small, using a raycast is most really fast and reliable. [insert raycast (pointer) references here] When some of these parameters change however, the raycast tool with a "laser-pointer" technique experiences more issues. In an environment that contains alot of objects in a small space, the error rate when trying to select an object rises.\cite{tools:ware1988using} This factor is multiplied when movement is added to the object (typical for games).

Multiple studies argue that there are better ways to perform object selection in a more complex and dynamic environment by tweaking this concept.\cite{selection:Argelaguet2008,interactions:Bowman1997} By using techniques that are designed for dynamic and cluttered environment the speed and error rate can be reduced. Two of these techniques are 'zoom' and 'expand'. On first selection the surrounding area of the selected object is enhanced to simplify the selection.

This tool is not without setbacks though. One problem that arises is the "lever-arm" problem which describes how an object thats selected with a raycast has limited number of possible manipulation actions.\cite{interactions:Poupyrev1996} Another big problem with ray-cast is trembling of the hand and twitches that occour when user tries to select an option. This has been given the name "Heisenberg effect" and is the cause of new interaction issues that arrives with this new medium\cite{selection:Bowman2001}:

\begin{itemize}
\item user dissatisfaction due to increased error rates,
\item discomfort due to the duration of corrective movements, which in the absence of physical support require an additional physical effort, and
\item unconfidence on which object will be selected after triggering the confirmation
\end{itemize}

\subsection{Selectors on objects}
\label{theory:toolsandtech:selector}
Strauss et al. presents a selector in his course notes\cite{tools:strauss2002design} as visualization of actions or operation to an object, in proximity to said object. The authors continues by listing benifits from utilizing selectors as listed below:
\begin{itemize}
  \item A number of different controls available to the user at all time
  \item Allows the user to isolate manipulation in 3D-space. This can limit the object to only be manipulated along the X-axis even though the user input is along all three axis'.
  \item It diplays the object that is manipulated and what operations are possible on an object at any time.
  \item The users attention stay on the object as the tools are in direct proximity to the active object.

\end{itemize}
