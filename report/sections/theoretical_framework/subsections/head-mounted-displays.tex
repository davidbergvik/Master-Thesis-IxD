\section{Head-mounted Displays}
The most common way of imerging yourself in a virtual environment is with a Head-mounted Display (HMD). This device is placed on the head of the user with separate views, one for each eye. The displays produce slightly different images which creates a stereoscopic view when percieved through the HMD. The experience is called stereopsis\cite{HMD:goldstein2016sensation}. This is achieved in two different way as of the current HMDs available in March 2017; Two different displays integrated into the headset that produces different images or one removable display that displays two separate images on the same screen. The second approach utilizes the steady rise of high performance smartphones with high resolution displays into the mobile market\footnote{https://deviceatlas.com/blog/16-mobile-market-statistics-you-should-know-2016}, as a smartphone can be used as a display and central processing unit for these HMDs.
HMDs can be abstracted based on their functionality, summarized by Bierbaum et al in their study about creation of virtual application using HMDs\cite{HMD:bierbaum2001vr}. Based on HMDs avaliable today, three classes of HMDs can be obtained using this way of abstraction based on interaction possibilities and performance;

\textbf{Low-level} HMDs has a removable display (mobile device) with internal tracking. Interactions are limited to the headset, no external interaction tools.

\textbf{Mid-level} HMDs has a removable display (mobile device) with internal tracking. Interactions extends to one remote interaction tool (controller) with internal tracking functionality.

\textbf{Hi-level} HMDs has two (2) internal displays, run on external machine. Interactions includes two controllers. Headset and controllers are tracked internally and with external sensors.
