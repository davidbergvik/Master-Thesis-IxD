\section{User-Centered Design}
\label{theory:user-centered}
According to Stanney et al. it is crucial to consider the human factors when designing successful applications in virtual reality environments. When the end user has little to no technical knowledge about the processes going on within the designed system or application, accounting for this is a valuable tool for a good user experience.
\cite{UCD:stanney1998human}
\subsection{Cognitive Load}
talk about this
\subsection{Flow}
\label{theory:user-centered:flow}
Flow is a term for describing a state of focus within creative work and was coined by Csikszentmihalyi in 1975 by studying playfulness, creativity and the characteristics of artists for years. \cite{UCD:boredom1975anxiety} The author claims that by studying pleasure we can move away from tedious tasks in a professional environment. Flow is a powerful way of exposing creativity when solving a problem or creating something, and the building blocks of flow is presented by Csikszentmihalyi as follows:\cite{UCD:csikszentmihalyi1996flow}

\begin{itemize}
  \item There are clear goals every step of the way
  \item There is immediate feedback to one's actions
  \item There is a balance between challenges and skills
  \item Action and awareness are merged
  \item Distractions are excluded from consciousness
  \item There is no worry of failure
  \item Self-consciousness disappears
  \item The sense of time becomes distorted
  \item The activity becomes autotelic
\end{itemize}
