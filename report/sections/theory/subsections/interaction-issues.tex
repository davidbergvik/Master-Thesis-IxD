\subsection{Problems with interactions in VR}
Fatigue is one of the biggest problems with VR [A survey of 3D object selection techniques for virtual environments].
Selection techniques are more severe on arm and wrist strain/pain while navigation can induce simulation sickness.

\subsubsection{Interaction Technique issues}

[New Directions in 3D User Interfaces] presents four ways that the majority of interaction techniques exhibit generality:
\begin{itemize}
  \item Application- and domain-generality: The technique was not designed with any particular application or application domain in mind, but rather was designed to work with any application.
  \item Task-generality: The technique was designed to work in any task situation, rather than being designed to target a specific type of task. For example, the design of the ray-casting technique does not take into account the size of the objects to be selected and becomes very difficult to use with very small objects (Poupyrev et al. 1997).
  \item Device-generality: The technique was designed without consideration for the particular input and display devices that would be used with the technique. Often techniques are implemented and evaluated using particular devices, but the characteristics of those devices are not considered as part of the design process. For example, the HOMER technique (Bowman and Hodges 1997) is assumed to work with any six-degree-of-freedom input device and any display device, but all of the evaluations of this technique have used a wand-like input device and a head-mounted display (HMD).
  \item User-generality: The technique was not designed with any particular group of users or user characteristics in mind, but rather was designed to work for a “typical” user.

\end{itemize}
