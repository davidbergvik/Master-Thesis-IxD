\subsection{Recent design principles for screen UI}
The design principles for the digital age has until the past couple of years relied to some extent on Skeumorphism to provide users with clues on the interface. Its use is explained by Norman [Design of everyday thing] as to designing a consistent conceptual model and by visualising the clues to support logical reasoning. Apple has for many years used design-principles connected to Skeumorphism with their software applications and operation system and became part of all areas of user interface design [Skeumorfism or flat design]. Later styles of skeuomorphism does however move towards mimicing past designs of a feature instead of the object that is used today, as Greif [What skeuomorphism is (and isn’t).] presented in 2012. One example of this is the floppy-disk icon that is frequently used as a design-pattern for saving or storing, even though floppy disks no longer are supported on modern computers. The long reign of the skeuomorphic design principles were however challenged by a flatter and more minimalsitic design which Microsoft embraced in 2013 with their new design ‘Microsoft Metro UI’. [Flat pixels: the battle between flat design & skeuomorphism]. This new design pattern has since then been embraced world wide and is still used in throughout the industry. This pattern however has its setbacks. The clean and subtle look might be pleasing on a new hi-def display, but its lack of content can lose the users emotional connection.[ The trend against skeuomorphic textures and effects in UI design]
